%\documentclass[11pt]{article}
\documentclass[print,ms]{nuthesis}
%\usepackage{times}

%\setlength{\textwidth}{6.5in}
%\setlength{\textheight}{9.0in}
%\setlength{\topmargin}{-.5in}
%\setlength{\oddsidemargin}{-.0600in}
%\setlength{\evensidemargin}{.0625in}

\newcommand{\secref}[1]{Section~\ref{#1}}
\usepackage{xcolor}

\usepackage[dvipdfm,%
bookmarks=true,%
bookmarksopen=true,%
bookmarksnumbered=true,%
bookmarkstype=toc,%
pdftitle={Derek Weitzel Thesis - Campus Grids},%
pdfsubject={},%
pdfauthor={Derek Weitzel},%
pdfkeywords={Weitzel, Grid, Campus},
%linktocpage=true,
linkbordercolor=white
]{hyperref}


%\newcommand{\doublespace}{\baselineskip0.34truein}
%\newcommand{\singlespace}{\baselineskip0.16truein}
%\newcommand{\midspace}{\baselineskip0.24truein}
%\newcommand{\midplusspace}{\baselineskip0.26truein}




%\usepackage{setspace}
\usepackage{algorithmic}
\usepackage{algorithm}

\usepackage{graphicx}
\usepackage{calc}
\usepackage{url}

\newlength{\imgwidth}

\newcommand\scalegraphics[1]{%   
    \settowidth{\imgwidth}{\includegraphics{#1}}%
    \setlength{\imgwidth}{\minof{\imgwidth}{\textwidth}}%
    \includegraphics[width=\imgwidth]{#1}%
}



%\doublespacing

%\author{Derek Weitzel\\
%Computer Science and Engineering\\
%University of Nebraska--Lincoln\\
%Lincoln, NE 66588-0115\\
%dweitzel@cse.unl.edu
%       }
       
\begin{document}
\frontmatter
\title{Campus Grids}
\author{Derek Weitzel}
\adviser{Dr. David Swanson}
\adviserAbstract{Dr. David Swanson}
\major{Computer Engineering}
\degreemonth{May}
\degreeyear{2011}

\maketitle

\begin{abstract}

It is common at research institutions to maintain multiple clusters.  These might fullfill different needs, different policies, or  represent generations of hardware.  Many of these clusters are under utilized while researchers at other departments may require these resources.  Linking these clusters with traditional grid software would require changes in security and execution environments.  In this paper we will describe a framework and technology that are used to link departmental clusters such that submission at one cluster could lead to execution on another.  We evaluate the framework on five important attributes found in campus grids.  This framework is then further expanded to bridge campus grids into a regional grid.

%the framework is designed as lightweight, and leveraging existing security infrastructure.   All of these components are functional and are running research jobs in production.

\end{abstract}

%\newpage
\tableofcontents
%\newpage

% \doublespace
\mainmatter
\chapter{Introduction}
\label{sec:Introduction}

%\subsubsection* {The problem we have solved}

%\begin{itemize}
%\item
%Concentrate on making {\em this} assertion and {\em only} this assertion in a
%succinct set of 1 to 3 paragraphs 

%\item Department clusters waste power by being under utilized for significant portions of time.
%\item Researchers have peaks in usage and need overflow capacity.
%\item Move single core jobs around to idle clusters, freeing up space for MPI jobs.
%\item Users want a single execution environment.  This is an expressed goal of the Condor project.
%\item Increased utilization of a cluster can reflect well on the department.
%\item Use existing security infrastructure.

%\item
%A common mistake is to explain too much of the problem context first. Instead,
%state the problem essentially as a claim, and leave explanations supporting
%your claim to the next part, ``Why it is not already solved.''

%\end{itemize}

A campus grid is a framework for distributing computation among independent clusters within a campus.  A campus typically consists of multiple compute clusters that are independently administered.  A campus grid's purpose is to increase the processing power accessible to users by connecting compute resources.   The compute clusters can benefit from increased utilization, while the users can benefit by increased processing power.  

The Holland Computing Center (HCC) created a campus grid that spans two clusters and can overflow onto national grid infrastructure.  The HCC grid borrowed concepts and techniques from earlier campus grids and a national grid.  The campus grid is quickly becoming the model of a homogeneous campus grid at other sites.

The HCC campus grid bridges clusters and the national infrastructure while running production processing jobs from on-campus researchers. Over the last X (TODO find stats on number of hours) we have ran Y hours on the campus grid, along with Z hours bridging to other campus grids.


Generally, users are provided a uniform interface to submit jobs to remote clusters inside their campus.  In many campus grids, the user manually submits to specific clusters.  In others, a layer of software distributes the computation among clusters in the grid. 



As the speed of innovation in computing increases, it is common for a campus to have multiple generations of clusters.  These clusters are commonly bought as independent systems, not meant to share work between them.  This causes users to flock to the newest generation hardware, hoping to increase the performance of their application.  This movement of users leads to under utilization of older hardware, and increased demand on the newest.

This is just one of many situations that would cause users to underutilize hardware.  Other situations could be:

\begin{itemize}
\item \textbf{Departmental clusters:}  Each department has a dedicated cluster.  This cluster specialization can cause under utilization when department researchers are not using the resources.
\item \textbf{Peaks of usage:} In general, users have peaks to their usage around deadlines.
\item \textbf{Parallel jobs:} While a cluster is draining nodes for a large parallel job, the cluster could fill those drained slots with pre-emptable usage.
\end{itemize}

One way to alleviate this demand is to move single core jobs to other clusters that have idle cycles.  Recently, the single core performance has been stagnant, therefore moving the single core jobs to the older hardware will not significantly decrease performance.  Also, by moving the single core jobs, it can free the newest hardware for large parallel jobs which can benefit from better interconnects, larger and faster storage, and increased core count that are on the newest hardware.




%\subsubsection* {Why the problem is not already solved or other solutions 
%are ineffective in one or more important ways}


%\item
%Your new idea need not solve every problem but it should solve at least one
%that is not already solved

There have been several attempts to create a distributed campus grid.  They all use some technology to distribute and schedule the jobs on the grid.  Some methods for distribution are commercial products such as Moab \cite{website:moabgrid}.  Others are translation layers between a generic description language and a scheduler, such as Globus \cite{foster1997globus}.  Still others are entire resource managers that can span multiple clusters like Condor.  Each of these solutions can be built to create a campus grid, but they all have drawbacks that are highlighted below.






%\item 
%A common solution to linking clusters is condor flocking.  Flocking requires every cluster to run condor daemons on their nodes.  

% Last case, since it's simalar to mine.
%Several universities have approached the problem utilizing condor flocking.

%\item
%This is the place to provide a succinct description of the problem context
%giving enough information to support the claim that a problem exists, made in
%the preceding problem declaration.





%\subsubsection* {Why our solution is worth considering and why is it effective
%in some way that others are not}

The framework described in this paper creates a production campus grid.  The campus grid includes technology to allow clusters to join the campus grid.  It integrates clusters that use several schedulers, and uses production quality software integrated into a solution that is deployed at several cluster in the U.S.

The campus grid framework provides a method for users to run jobs transparently on distributed resources.  The jobs are sent to available resources on distributed clusters on the campus grid.  Further, it can expand beyond the campus and onto other campuses by simple configuration changes.

The campus grid is used in production at the University of Nebraska, connecting two geographically separate clusters.  Users utilize the campus grid by submitting to one connected clusters, while their jobs run on either cluster.  The jobs are transported with their input files to a execution sandbox on the worker node.  Further, the campus grid expands beyond Nebraska, reaching Purdue and Clemson university.


% This will largely be from CHEP paper
\section{Attributes of Campus Grids} \label{sec:attributes}
In this section, we explore five characteristics of HTC-centric campus grids. While the list isn't exhaustive, we've 
found campus grids can be characterized by how they approach trust relationships, job submission, resource 
independence, accounting, and data management.


\subsection{Trust relationships}
A successful campus grid must have an acceptable trust model in order to succeed.  A trust relationship enables 
a resource provider to grant campus users controlled access to the resource, and may be established through 
 sociology and/or technology-based security methods.

In the OSG, the trust model used is designed to be homogeneous and to meet the most stringent requirements of 
all participating sites.  The implementation involves using Globus's Grid Security Infrastructure (GSI) with VOMS 
attributes a PKI extension \cite{farrell2002rfc3281}.  The GSI model is widely 
accepted, allowing the OSG to participate in the Worldwide LHC Computing Grid (WLCG).  Fermigrid's campus grid user 
authentication is based on the GSI model.  While it provides a highly secure, decentralized authorization model and 
proven at the worldwide scale, it is more difficult for end users compared to traditional username/password 
authentication.  Thus, campus grids may be motivated to use alternate trust models.

On-campus resource providers may have a higher degree of trust than at the national level due to 
sociological reasons.  This trust may just be based on locality -- it is easier to establish a working relationship 
with a colleague locally on campus than 1000 miles away.

A technical reason for different trust relationships between campuses and larger grids is the location of user 
job submit hosts.  Unlike the OSG, where users can submit jobs from any worldwide host, campus users often submit from 
a few trusted campus resources.  If limited to a few well-managed hosts, IP-based security may be sufficient for 
campuses, as the security is applied to submit hosts rather than cluster entry points.

Security requirements on some university campuses are simply less stringent than that of federal labs, explaining 
Purdue and Wisconsin's preference for IP security compared to FermiGrid's GSI.  A campus may 
not have strict policies governing user job separation or traceability requirements.  Some campus clusters may be 
satisfied with running any job originating from elsewhere on the campus to an unprivileged account.  When a job crosses 
domains (from local cluster to across campus, or from campus to the national grid), it must 
satisfy the security requirements for the destination domain.  Thus, if a campus grid would like to bridge to the 
national grid, users most be able to associate GSI credentials with their jobs.


\subsection{Job submission}
In order for a HTC-oriented campus grid to function, users need a usable job submission interface.
The Globus Toolkit \cite{foster1997globus} provides the GRAM interface for job 
submission and corresponding clients.  GRAM layer abstracts above the batch system; the 
user interacts with GRAM, and GRAM converts these actions into batch system commands at the destination.
The GRAM interface is used by the OSG, and is being used at the scale of over 100 million jobs a year.  The GRAM 
interface abstracts many batch system constructs, and is also used 
on the TeraGrid to submit larger jobs running on hundreds or thousands of cores.  While GRAM can be used directly, 
users almost exclusively prefer to interact with it via Condor-G  \cite{frey2002condor} , which provides a batch 
system interface on top of GRAM.  Fermigrid relies on Condor-G submission to GRAM for job submission.

An abstraction layer like GRAM introduces a new user experience (even if Condor-G is used), requiring new expertise. An 
alternate approach is to use batch system software that can interact with multiple instances of itself.  By 
linking resources at the batch system level rather than adding an abstraction layer on top, we improve the user 
experience - users no longer need to learn additional tools.  The tools do not need to translate errors across 
different domains, easing a common source of frustration in the grid.  When Condor-G is used, we have a batch-system 
interface abstracting an API which, in turn, abstracts remote batch systems - error propagation is extremely difficult.  
In the Purdue, GLOW, and HCC campus grids, 
resources are linked through use of a common batch system, Condor, through a mechanism Condor refers to as 
``flocking".  A hybrid between Condor-only and GRAM is given by GlideinWMS \cite{sfiligoi2008making}.

In our observations, the closer the grid user experience is to the batch system user experience, the more likely a 
user will adopt the campus grid.



\subsection{Resource Independence}
Compared to a corporate IT environment, one unique aspect of universities is the diversity of management of 
computing 
resources.  On a campus, several distinct teams may manage distinct clusters due to campus 
organization or ownership.  Management of resources may be divided by college or departmental level.  One 
characteristic of campus grids is thus the independence of resources - the level of decision-making delegated out to 
the resource providers.

The simplest campus grids can be formed by requiring all clusters on campus to run the same batch system and linking 
batch system instances - GLOW's use of Condor is an example.  Every cluster in GLOW runs the Condor batch system, 
providing a common interface.  System administrators are not free to chose their own batch systems if they want to 
participate in this grid (participation is voluntary, and participants obviously believe the benefits of GLOW 
membership outweighs this drawback).  It may be desirable for a specialized cluster to have a distinct batch system  
from the rest of the campus; resource independence allows the cluster owners to best optimize their resource to suit 
their needs.

Resource independence comes at a cost to the end-user.  Extremely heterogeneous resources can be difficult to 
integrate at the software level - a binary compiled for Linux will not be compatible with Windows.  Some guarantees 
about the runtime environment or other interfaces need to be clearly articulated to prevent frustration.  Differences 
that are unavoidable or are expected to be handled by the user should be clearly expressed to the user \cite
{raman2002matchmaking}.  At the OSG level, we have found the users often frustrated by the amount of heterogeneity, 
especially compared to using a single site or a grid with a smaller number of sites \cite{zvada2010cdf}.


\subsection{Accounting}
Accounting may not seem to be an important grid characteristic -- it certainly isn't required for 
users to successfully run job.  However, it is critical for the long term health as it provides a quantitative 
measurement of the grid's value.  Accounting is also required for resource and users to ``barter" computing hours, one  
economic model for the grid.

Accounting systems do not need to be technically advanced.  Most batch systems provide a local accounting 
system.  The most basic method is for each cluster to parse these logs into a CSV file per cluster, and to 
build an Excel spreadsheet out of the aggregated files.  This is functional, but painful when statistics are needed 
more than once a month.  Most batch system vendors sell accounting systems usable for multiple clusters, provided all 
clusters involved use the same batch system.

Many research computing centers have written their own accounting systems at some point; most implementations are in 
the style of  PHP-based web interface on top of a custom database, again fed by custom log-parsing scripts. Both the 
OSG and TeraGrid have spent effort on accounting software to suite their needs.  The OSG�s, Gratia \cite{gratiaweb}, is 
designed to be reusable by other organizations, and is in use by all four campus grids�discussed in this paper.

Any site-local accounting systems -- homegrown, vendor provided, or designed for the national 
grids -- can work at the campus grid level as long as they can answer the following questions for a given time 
period:
\begin{itemize}
\item How much computing resource was consumed overall?
\item How much computing resource did a specific user/group consume?
\item How much computing resource did a specific user/group consume on resources they did not own? I.e., 
�How much did I get from resource sharing?�
\item How much computing resource did a specific cluster provide?
\item How much computing resource did a specific cluster provide to groups that did not own it?  I.e., �How 
much did I give away due to resource sharing?�
\end{itemize}




\subsection{Data Management}
Scientific data management presents two challenges for research computing centers: volume of data and archival 
requirements.  The data volume is often larger than a single scientist can keep on his personal systems, and 
archiving requires expertise outside his field.

Distributed computing can present an additional challenge: managing data location.  Data access costs may be 
variable between different resources on a grid, or required data may simply be unavailable at some locations.  A simple 
solution is to export the same file system to all resources, hiding data locality from the user.  Unfortunately, this 
solution breaks down outside the campus and may break down in highly-distributed campuses.

More complex solutions include declaring data dependencies for jobs explicitly inside the job 
submissions (gLite WMS \cite{andreetto2008glite}, Condor), promoting data to be a top-level abstraction like jobs (Stork \cite
{kosar2005stork}), or promoting data to be the central concept instead of jobs (iRODS \cite{irodswebsite, 
rajasekar2007irods}).  The CMS model separates the data management and job submissions systems, allowing the job 
submissions to simply assume all data is available (CMS PhEDEx \cite{phedex}).

While any system can be used for campus grids, the examples we consider in Sections \ref{sec:others} and \ref
{sec:hcc} either export a file system or utilize the tools from the job submission system.  The addition of a separate 
data management system often presents such complexity to the users that not buying more hardware and not using 
distributed computing is the cheaper alternative.  However, we note iRODS is an increasingly popular option and may 
have significantly decreased the operational cost of data management systems.


\section{Background}
\subsection{High Throughput Computing}
High Throughput Computing (HTC) is defined as tasks that require as much computing power (throughput) as possible over long periods of time \cite{gridbook-htc}.  This is in contrast to High Performance Computing (HPC), where users are concerned with response time.  This style is typified by ensembles of independent single processor jobs with no communication between them.  This is not to say there isn't coordination, many workflow managers can utilize HTC to solve complex problems with many steps.  

As there is no parallel communication between the jobs in a given task, they can be distributed across multiple resources.  This increases 
throughput, the end-goal of HTC.  HTC can use pooled resources mostly interchangeably and as such is well suited to 
distributed and grid computing models.  The OSG has demonstrated its technologies are successful; in Q4 2010, the OSG 
averaged over a 400,000 jobs and a million computational hours a day using HTC.


\subsection{Condor}

\subsection{Open Science Grid}


\subsection{Campus Grids}



Building a campus grid represents a significant commitment for both users and resource providers, which should be 
evaluated against the benefits.  The primary benefits are:
\begin{itemize}
\item \textbf{Resource sharing}: An HTC-based approach focusses on using all resources effectively.  Resources are 
typically bought for peak, not average, usage; integrating the idle time across the entire campus and using it improves 
the value of the investment.
\item \textbf{Homogeneous interfaces to multiple resources}: Moving researchers from one resource to another results in 
a (possibly large) upfront cost in time and energy.  By providing a homogeneous interface across the campus, 
researchers can quickly utilize new resources without the pain of migration.
\item \textbf{Independence from any single computational resource}:  It is expensive to provide highly available 
resources.  If do not rely on a specific single cluster, individual cluster downtimes have a smaller impact.  This 
reduces the need for high levels of redundancy and stretches the campus computing budget further.
\end{itemize}

An obvious requirement for campus grids is having multiple resources on campus.  However, this is not sufficient for 
resource providers - the resources should also be interchangeable.  A grid composed of a single AIX cluster, Linux 
cluster, and Windows cluster, will likely never see any resource pooling or sharing.  This does not necessarily imply 
the resources need to be identical - complete homogeneity is typically impossible due to individual resource 
requirements or ownership.

A mistake in campus grids is to focus on the infrastructure for pooling resources without similarly engaging and 
supporting the activities of users.  An analogy can be made to flows: if there is a sink (resources) with no source 
(user jobs), the flow quickly stops, and the campus grid is forgotten.  Personnel investment must be made to engage 
the user community.  Even before this investment is made, a campus should identify whether the on-campus scientific 
computing has a significant portion of tasks that can be converted to HTC workflows.  Prioritization should be applied 
so the users with the simplest workflow and the most to benefit are converted first.  Tasks with the following 
characteristics should typically be avoided:

\begin{itemize}
\item Large scale (multi-node) MPI, as these require specific tunings to the given resource.
\item Multi-day jobs.  Shared resources often need to be reallocated quickly back to the owner, meaning these jobs are 
unlikely to complete.
\item Sensitive data or software.  Tasks with sensitive (for example, HIPPA-protected) data may have legal boundaries 
preventing distribution.  Software with strict licenses may also be illegal to use across the grid.
\end{itemize}



%\begin{itemize}
%\item
%A succinct statement of {\em why} the reader should care enough to read the
%rest of the paper.
%\item The framework described in this paper is designed as a modular framework that will allow clusters to overflow onto each other and to the grid.  This framework requires running condor on only one node in the cluster.  Each job that comes from another cluster will go through the default scheduler, whether it's PBS, SGE, or LSF.

%\item
%This should include a statement about the characteristics of your solution to
%the problem which 1) make it a solution, and 2) make it superior to other
%solutions to the same problem.

%\end{itemize}


%\subsubsection* {How the rest of the paper is structured}


The rest of this paper first discusses related work in
\secref{sec:RelatedWork}, and then describes our implementation in
\secref{sec:Implementation}. \secref{sec:Evaluation} describes how we evaluated
our system and presents the results. \secref{sec:Conclusion} presents our
conclusions and describes future work.


\chapter{Related Work}
\label{sec:RelatedWork}

%\subsubsection*{Other efforts that exist to solve this problem and why are they
%less effective than our method}

\section{Technology to create a campus grid}

%\item
%Resist the urge to point out only flaws in other work. Do your best to point
%out both the strengths and weaknesses to provide as well rounded a view of how
%your idea relates to other work as possible

\subsection{Globus} \label{sec:globus}
Globus is a translation layer between the (globus specific?) Resource Specification Language, and the local resource manager.  It has been very successful in that is has a large install base.  Globus also has deep integration with standard grid credentials such as PKI.

Placing Globus gatekeepers on each cluster would allow jobs to be submitted to each cluster without modifying the underlying batch system.  But, this would require a higher layer of abstraction over the Globus gatekeepers to optimally balance load between clusters.  Additionally, Globus has well known limitations such as job a low submission rate, and high resource usage.  Additionally, Globus implements GSI (certificates) security that is inconstant with most existing campus security architectures.  Also, it does not provide a method for transparent execution on other clusters, which experience on the OSG has shown is important to users.

\subsection{Condor Flocking} \label{sec:flocking}
Another single vendor solution is Condor.  Each resource can run Condor on their clusters and 'flock' \cite{epema1996worldwide} to each other.  In this solution, jobs would be balanced on each resource due to Condor's greedy scheduler algorithm.  

Flocking jobs is accomplished by a multi-step process.  First, the \texttt{condor\_schedd} reports to the remote \texttt{condor\_collector} that it has idle jobs available to run.  During it's next iteration, the remote \texttt{condor\_negotiator} contacts the \texttt{condor\_schedd} to match available resources to the requested jobs.  When the \texttt{condor\_schedd} receives a match, it contacts the resource directly to claim it.  After the claim is successful, the job starts on the remote resource.

Condor flocking has many advantages.  Since the job submitter (\texttt{condor\_schedd}) directly contacts the executing resource, there is no additional load on any central node.  Flocking handles failures gracefully.  If an execution resource becomes disconnected, Condor will attempt to reconnect, or re-run the job elsewhere.  Jobs will still execute on previously claimed resources if the central manager becomes disconnected.

But, this solution again requires each resource to run Condor as their scheduler and resource manager.  Additionally, Condor must be running on each worker node, increasing the administration requirements.  This suffers from the same problem as Moab, restricting further innovation to the confines of a single solution.



\subsection{GlideinWMS}
GlideinWMS

\subsection{Panda}
Panda

\subsection{Moab}
One solution to build a grid is to use a single vendor/software solution.  For example, Cluster Resources offer a solution Moab Grid Suite\cite{website:moabgrid}.  This solution requires each resource to run a single piece of proprietary software, Moab.  Moab is a meta-scheduler, using PBS to manage the underlying resources.  By using Moab, the development of new grid technologies are limited to what can be done in Moab.






%\item
%In a social and political sense, it is {\em very smart} as well as ethical to
%say good things, which are true, about other people's work. A major motivation
%for this is that editors and program committee members have to get a set of
%reviews for your paper. The easiest way for them to decide who should review it
%is to look at the set of references to {\em related work} (e.g.,
%\cite{ARJ:95,BHR:90,Go:97}) to find people who are likely to be competent to
%review your paper.  The people whose work you talk about are thus likely to be
%reading what you say about {\em their} work while deciding what to say about
%{\em your} work. 

%\item
%Clear enough? Speak the truth, say what you have to say, but be generous to the
%efforts of others.



\section{Other Campus Grids}
\subsection*{Other efforts that exist to solve related problems that are
relevant, how are they relevant, and why are they less effective than our
solution for this problem}

\subsection{Virginia Campus Grid}
The Virginia Campus Grid \cite{humphrey2005university} designed a campus grid using the Web Services Resource Framework (WSRF) with Globus.  The goal of the campus grid was use as much existing infrastructure as possible.  The grid utilized the existing authentication system by developing a new credential generator called CredEx \cite{del2005credex} that interacts with the local LDAP servers to create PKI certificates.  Globus version 4 (now depreciated) was used to interact with the Linux clusters on campus.

Another focus of the Virginia campus grid was policy expression and enforcement.  This is a common theme for many grids since they span multiple administrative domains.  In the virginia grid, a enforcement service would enforce these rules by cutting off and redirecting users to and from resources.  The load balancing would be enforced by the enforcement services, as well as policies regarding a resources preference for jobs.  Additionally, a broker was developed to distribute jobs.

The Virginia campus grid has many attributes shared with other Campus Grids.  The Virginia grid approaches trust relationships by utilizing the existing campus authentication infrastructure.  Job submission is handled by WSRF and distributed with a custom developed broker.  Resource independence is not discussed in the paper, but there are many central services such as the enforcement service and the broker.  A downtime in either of these services would limit usefulness the grid.  Accounting is handled with the central IT along with authentication.  Data management is not addressed in the campus grid.  Data management is very important to any campus grid given the rising demands of data intensive computing for scientific applications.


\subsection{Oxford Campus Grid}
The Oxford Campus Grid \cite{wallom2006oxgrid} built a comprehensive campus grid including both compute and data provisioning.  The compute provisioning used Condor-G \cite{frey2002condor} and a information server.  The information server injected resource specific information into the Condor-G matchmaker, allowing Condor to match jobs to appropriate resources as well as follow resource policies.  For data provisioning, the Storage Resource Broker (SRB) \cite{baru1998sdsc} was used with a dedicated data vault.  Authentication is handled through on-campus kerberos.  Accounting is done by a custom daemon written at Oxford that keeps detailed statistics for every job.

The Oxford campus grid very closely resembles the Open Science Grid model.  Each resource has a gatekeeper, a central node that provides access to the underlying nodes.  The information server and virtual organization management both have analogies in the OSG.  
The resource broker and data vault conflict with the design of the OSG.  Both of these resources are single points of failure that can severely degrade the usability of the campus grid.  


\subsection{GLOW}
GLOW \cite{gridworkshopweb, glowwebsite} is an University of Wisconsin grid used at the Madison campus to distribute 
jobs on their all-Condor grid.  Security is based on IP whitelisting.  Since all resources are based on Condor, job 
submission and distribution is managed through the same Condor-only mechanisms as Purdue.  While there is a central 
team available to assist with management, each resource is free to define its own policies and priorities for local 
and remote usage.  Cluster ownership is distributed, although there's also a general-purpose cluster available.  
Software and data is managed by an AFS \cite{morris1986andrew} install and Condor file transfer.

\subsection{Purdue}
The Purdue campus grid \cite{smith2008implementing, gridworkshopweb} is part of a larger grid, Diagrid, which serves a 
number of smaller 
universities in Indiana and  Wisconsin.  This grid is based upon the Condor and Condor flocking technology.  All jobs 
are submitted via Condor.  For security, Purdue manages a small number of submit hosts that are allowed to run jobs on 
their grid.  External jobs can flock to Purdue and are mapped to an unprivileged user on the execute host.  In order to 
maximize the resources in its grid, Purdue also installs Condor on its PBS-based clusters.  Each batch system makes 
decisions independently, except any PBS job on a given node will preempt any Condor jobs.  While idle resources are 
thus utilized, PBS may unnecessarily interrupt Condor jobs and all Condor jobs are inherently lower priority.  The 
largest resources are centrally administered by a single organization, but there are large pools independently 
configured and managed.  Usage accounting is done through Condor and a homegrown system.  On large subsets of the grid, 
data is kept on a shared file system but no single file system is exported to all resources.  Condor file transfer can 
be used throughout the grid.

\subsection{FermiGrid}
Fermigrid \cite{gridworkshopweb, chadwick2008fermigrid} is made up of resources located at the Fermi National 
Accelerator Laboratory in Batavia, IL.  The Fermigrid campus grid is the closest example found of a ``mini-OSG".  Its 
uses the same CE software, information systems, and storage elements as the OSG.  Trust relationships on Fermigrid are 
based on the Grid Security Infrastructure (GSI) \cite{farrell2002rfc3281}, the same authentication method used by 
OSG.  Job submission is managed by Condor-G through a Globus submission layer to the clusters.  This same method can be 
used to submit to OSG, providing one strategy to getting users from campus to the national grid.  Some clusters are 
managed by a central team, while others are done independently.  Some of the grid services (authorization and 
information services, for example) are run centrally.  Accounting is done through Gratia \cite{gratiaweb}, the same 
software that is 
used on the OSG.  A central cluster file system is available to most clusters, but Globus-based file transfer is also 
heavily used.

%\item 
%Many times no one has solved your exact problem before, but others have solved
%closely related problems or problems with aspects that are strongly analogous
%to aspects of your problem



\chapter{Implementation}
\label{sec:Implementation}

\subsubsection*{What we (will do $|$ did): {\em Our Solution}}
\begin{itemize}

%\item   Another way to look at this section is as a paper, within a paper,
%describing your implementation. That viewpoint makes this the introduction to
%the subordinate paper, which should describe the overall structure of your
%implementation and how it is designed to address the problem effectively.

%\item   Then, describe the structure of the rest of this section, and what each
%subsection describes.

\item
Created a campus grid integrating 3 clusters on a campus into a grid.  Submission to any of the clusters could overflow to the other 2.

\item
Overflow to the grid

\item
Using offline ads to efficiently match jobs to glideins on the non-condor cluster.

\end{itemize}



\subsubsection*{How our solution (will $|$ does) work}
%\begin{itemize}
%\item   This is the body of the subordinate paper describing your solution. It
%may be divided into several subsections as required by the nature of your
%implementation.

%\item   The level of detail about how the solution works is determined by what
%is appropriate to the type of paper (conference, journal, technical report)

%\item   This section can be fairly short for conference papers, fairly long for
%journal papers, or {\em quite} long in technical reports. It all depends on the
%purpose of the paper and the target audience

%\item   Proposals are necessarily a good deal more vague in this section since
%you have to convince someone you know enough to have a good chance of building
%a solution, but that you have not {\em already} done so.

\textbf{ Campus Factory is the heart of the operation.  It provides all throttling and logic to the submit of glidein jobs to the non-condor cluster.}

\section{Glidein Jobs}
Glidein  \cite{frey2002condor} is a pilot job based grid submission that creates an overlay network.  Glidein is designed to used standard Condor mechanisms to advertise it's availablity to a Condor Collector process, which is queried by the Scheduler to learn about available resources.  Each user job is run in a sandbox on the local disk and is provided with a consistent execution environment across hosts and clusters.

Pilot jobs are used by many physics experiments \cite{nilsson2008experience, zvada2010cdf, bradley2010use}.  Physics experiments create pilot frameworks because they have need for very large distributed workflow management systems.  Pilot workflow management systems are commonly used for multiple reasons:
\begin{itemize}
\item \textbf{Scheduling Optimization:} The pilot reports only when a cpu is immediately available to run a job.
\item \textbf{Input/Output Automation:} The pilot can transfer input and output seamlessly for the user.
\item \textbf{Monitoring:} The monitoring of a job can be more accurate while the job is running.
\item \textbf{Fault Tolerance:} A job failure can be more accurately detected and recovered.
\item \textbf{Multiple Runs:} Each pilot can run multiple jobs serially, reducing submissions through the site gatekeeper.
\end{itemize}

The factory uses glideins on non-condor resources in order to create a overlay condor cluster that can execute remote jobs via flocking.  The glideins report to a Collector unique to each Cluster and are configured to exit after a configurable amount of time. 

The glidein job requires six condor daemons packaged with a wrapper script.  When the job starts, the Glidein job will:

\begin{enumerate}
\item Create a temporary directory on the local disk.  This will be used for the job sandboxes.
\item Unpackage the glidein executables into the local disk.
\item Set the late binding environment variables for Condor to point to the temporary directory.
\item Start the  \texttt{condor\_master} daemon included in the glidein executables.
\end{enumerate}

The \texttt{condor\_master} will start the \texttt{condor\_startd} which will advertise itself to the glidein collector, therefore making the node available for remote jobs.  

Glideins are being used in production in the Open Science Grid using the software GlideinWMS \cite{sfiligoi2008glideinwms}.  They provide several advantages over regular job submission.  Each glidein sandboxes and monitors the user jobs it runs.  The glidein can run multiple user jobs inside a single Local Resource Management (LRM) job, and will continue to run until the configured time to stop.  Glidein host connects directly with the submitter host, removing the need for staging data.




\section{Campus Grid Factory}
\begin{figure}[ht]
\centering
\scalegraphics{images/FactoryOverview.pdf}
\caption{Overview of Campus Factory function}
\label{fig:campusfactoryoverview}
\end{figure}
The Campus Factory is a daemon that runs on non-condor clusters in order to submit glideins when additional resources are requested.  The Condor instance that the factory communicates with must be on a `gateway' node: Able to talk to both the remote clusters and the local nodes.   The Factory communicates with the \texttt{condor\_collector} daemon in order to detect requests for resources, and the \texttt{condor\_schedd} daemon to submit jobs to the LRM.  

The factory is an integral part of the campus grid because it allows non-condor clusters to participate in the grid.  A Condor cluster would not need the factory.  The factory submits glideins to the Local Resource Manager (LRM), allowing them to run with the priorities set in the LRM.  The jobs, once started, report back to the factory collector as a regular condor pool node.  The factory collector then is able to route jobs from other clusters to these available glideins just as it would for an all-condor cluster.

The Campus Factory is a python daemon that runs as a persistent condor job.  Since it runs as a condor job, the condor daemons will ensure that it stays alive, eliminating the need to monitoring an additional daemon.  The Campus Factory functions are shown in Figure \ref{fig:campusfactoryoverview}.  The factory will periodically query the collector for idle slots, and query the schedd for idle jobs.  The factory will talk to schedds on other clusters listed in the condor configuration variable \texttt{FLOCK\_FROM}.

The factory depends on Condor daemons to carry out many tasks such as:
\begin{itemize}
\item Run and maintain the factory daemon.
\item Submission of the glidein jobs to the LRM (See Section \ref{sec:condorandblahp}).
\item Collect and advertise information on the glidein jobs.
\item Negotiate with submitters in order to route jobs to glideins.
\end{itemize}


\subsection{Determining when to submit glideins}
The factory decides whether submit glideins to the underlying non-condor cluster.  The factory has 2 configuration options relating to submission of glideins to the LRM: \texttt{MaxIdleGlideins} and \texttt{MaxQueuedJobs}.

\begin{description}
\item[ \texttt{MaxIdleGlideins}] \hfill \\
An integer representing the number of idle slots that will be allowed before the factory stops submitting jobs. 

\item[ \texttt{MaxQueuedJobs}] \hfill \\
An integer number of queued glideins that will be allowed to be idle in the LRM before submitting more. 

\end{description}

The factory submission logic is as follows:

\begin{algorithm}
\begin{algorithmic}
\IF {idlejobs $<$ MaxIdleGlideins \&\& queuedglideins $<$ MaxQueuedJobs}
	\STATE $toSubmit \gets$ min(MaxIdleGlideins - idlejobs, MaxQueuedJobs - queuedglideins, idleuserjobs)
\ELSE
	\STATE $toSubmit \gets 0$
\ENDIF
\RETURN $toSubmit$

\end{algorithmic}
\caption{Algorithim for determining how many glideins to submit.}
\end{algorithm}

Additionally, the factory has logic to detect glideins that are not reporting to the collector.  This can happen when the the LRM cannot transfer files, if the BLAHP is incorrectly reporting the status of a job, or if there is something wrong with the glideins jobs.

\section{Flocking}
Flocking is enabled between the condor clusters, and the non-condor clusters.  The non-condor clusters run the condor daemons on one node that will handle flocking and submission of glidein jobs.  This node can also serve as the local condor queue for users to submit to. 

\section{Condor \& BLAHP}
\label{sec:condorandblahp}
Condor \& BLAHP \cite{blahp} provides the interface to the local batch system.  It is developed as a part of glite, and is included in the standard condor distribution. 

OfflineAds are used to efficiently match jobs to potential slots on a machine.  These were developed because we do not always want ~5 idle jobs running on a cluster.  We would rather have a few jobs submitted per day, and only submit more glideins if jobs will run there.


\section{OfflineAds}
Offline ads are designed to be used for power management. When a node hasn't been matched for a configurable amount of time, the machine can be turned off to save power. When the machine is preparing to turn off, it sends an offline ad to the collector that describes the machine. The OfflineAd includes all of the features that can be used when matching against an online resource.  When the offline ad describing the machine is matched to a job by the Condor negotiator, the negoiator inserts a new attributed into the offline ad called MachineLastMatchTime.  When using power management, the Condor Rooster periodically queries the collector for the offline ads, and wakes the powered off matched nodes to run the job.

The collector maintains the offline ads for a configurable amount of time using the configuration variable OFFLINE\_EXPIRE\_ADS\_AFTER. During this time, the condor negotiator will treat the offline ad just as it would a real ad and match it to idle jobs. Since the Negotiator sees no difference between running and offline ads, the offline ads will be matched even when flocking from another condor pool.  This is how the offline ads are used in the campus factory.

\subsection{Creating OfflineAds}
The factory detects classads of running glideins, copies the classads, and re-advertises them as offline ads. Since the offline ad is an exact copy of a running glidein, it is reasonably expected that you can get a similar glidein when you submit to the local scheduler.

To change a glidein classad into a offline ad, the following attributes must be changed

\begin{itemize}
\item Offline = true
\item Name - to a unique name
\item MyCurrentTime = LastHeardFrom - Time now
\item ClassAdLifetime - To address a bug in the handling of offline ads by Condor. This is how many seconds the collector will keep this ad
\item State = Unclaimed - Make sure it will match with idle jobs.
\item Activity = Idle - Again, for matching 
\end{itemize}

Convenience attributes: 
\begin{itemize}
\item PreviousName - Value of 'Name' attribute of the original ad. Useful for debugging. 
\end{itemize}

\subsection{Managing OfflineAds}
By default, the factory will attempt to maintain a specific number of offline ads. This can be done in a number of ways, such as always maintaining the newest 10 ads. Or you can sort by different 'types' of machines (big memory, big disk), and keep an assortment of unique ads.  Currently, the factory only keeps the latest 10 ads.

To maintain the classads, the factory queries the collector for offline ads. If it detects more than 10, it will only keep the newest 10. If less then ten, the offline ad manager will list the site as Delinquent, and will recommend submitting more glideins.

\subsection{Influence OfflineAds have on the Factory}
The OfflineAds do not completely replace all logic relating to the factory. The site must still meet the idle glideins/idle slots requirements that were originally used to throttle new glideins. The OfflineAds do replace the need for the factory to query remote schedd's to for idle jobs. The negotiator and collector take care of all job matching with semi-accurate glidein classads.

The logic of the factory follows:

\begin{enumerate}
\item Are there idle startd's?
\item Are there idle glideins in queue?
\item Have any of the idle offline ads matched in the last x seconds?
\item If we get this far, submit some glideins. 
\end{enumerate}


\section{Holland Computing Center campus grid} \label{sec:hcc}
In designing a campus grid for the Holland Computing Center, we attempted to meet three goals:
\begin{itemize}
\item \textbf{Encompass}: The campus grid should reach all clusters managed by HCC.
\item \textbf{Transparent execution environment}:  There should be an identical user interface for all resources, whether 
running locally or remotely.
\item \textbf{Decentralization}: A user should be able to utilize his local resource even if it becomes disconnected from 
the rest of the campus.  An error on a given cluster should only affect that cluster.
\end{itemize}

Like Purdue and GLOW, we have based the campus grid upon Condor.  Each resource has a Condor-based interface, giving an 
identical experience regardless of what the user considers his or her ``local" cluster.  While two of the local 
clusters run Condor as the primary batch system, one is based upon PBS.  PBS was chosen because of its superior 
scheduling of large-scale MPI jobs required for that resource.  GLOW's Condor-only approach did not fit our case, and 
Purdue's model of running multiple schedulers was rejected because we wanted a less-invasive approach and because we 
wanted more efficient scheduling.  So, for integrating our PBS cluster, we developed the Campus Grid Factory (CGF) to 
provide a Condor interface for the non-Condor clusters; this is covered in Section \ref{sec:cgf}.

To enable decentralized operation, we utilized Condor flocking \cite{epema1996worldwide} between clusters.  Condor 
flocking enables a transparent execution environment by 
imitating the interaction between submit and execute hosts when talking to remote resources.  The jobs will continue 
to be managed by the original user's submit host, but the execute hosts can be outside what is managed by the local 
Condor pool.  Furthermore, flocking can handle communication errors between remote hosts and can recover, 
providing disconnected operation when resources are unreachable.

Through Condor flocking and the CGF, we have successfully encompassed all local resources.  To provide even more value 
to HCC, jobs can also bridge to the OSG and other campus grids.  The interface to the OSG uses the GlideinWMS \cite
{sfiligoi2008GlideinWMS, sfiligoi2008making} frontend software, while we link to other campuses using Condor flocking 
(the same method Purdue uses to link the campuses of DiaGrid).  Unlike Condor-G, which provides a Condor interface to 
GRAM, these two methods give the same user experience as using Condor as a batch system.

All clusters on the campus grid are managed by the Holland Computing Center, therefore the trust relationship 
between the hosts are implicitly strong.  A special account is set aside on the PBS cluster for the CGF to run 
campus grid jobs.  On the Condor-managed cluster, campus grid jobs run as user \texttt{nobody} while locally-submitted 
jobs run as the submitting user.

While all resources are run by the same team, we have the ability to provide distinct user priorities per resource 
through Condor.  Further, because Condor runs {\textit inside} PBS rather than alongside it, PBS can schedule its jobs 
without interrupting Condor ones.  An administrator can prioritize jobs submitted directly from the local cluster over 
those from a remote submission, even for the same user.

Each submission host runs the Gratia accounting software to provide user accounting.  Gratia was chosen because it has 
a clean separation between remote clusters and the central database (updates are done via HTTP), ability to integrate 
new resource types easily, and its ability to integrate into the larger OSG accounting.  For integration in the OSG, 
we have extended the software to record both the submission host and the remote OSG cluster utilized.

HCC does not have a shared filesystem across all clusters, so campus grid data management is handled by Condor file 
transfer.

\section{Bridging Campus Grids}
There are two methods for expanding the campus grid described in Section \ref{sec:hcc}: through GlideinWMS to the OSG, 
or by linking campus grids through flocking.  The benefits of bridging externally are obvious - increased throughput 
for the local user's jobs.  HCC has been able to utilize over 7,000 remote job slots at a time, and has been able to 
bridge to all the campus grids discussed in Section \ref{sec:others}.  We connect to FermiGrid and GLOW through the OSG and 
to Purdue via Condor flocking.

Unlike the OSG, where the trust relationship is well-defined and common between sites, trust is established between 
campuses with flocking on a case-by-case basis.
The current model for trust is based on limited trusted hosts.  Each site publishes a list of submit and negotiator 
hosts that are trusted to submit and accept jobs, respectively.  This implicitly trusts an entire campus, while the OSG 
trust model is based on virtual organizations that may have no relationship to a physical campus or submit host.

The full campus grid architecture, with bridging, is shown in Figure \ref{fig:campusgrid}.  In this campus grid, the 
user first submits jobs to the local condor cluster (1).  If the local cluster can fulfill the user's needs, then the 
jobs will remain there.  If the local cluster is full or cannot meet the user's demand, Condor flocking will start 
submitting jobs to other campus clusters (2), either Condor or CGF-based.  If the on-campus 
resources are unable to meet the user's request, the local Condor schedd will expand its reach again (3) by looking 
outside the campus.  The jobs can also be sent to the OSG via flocking to a GlideinWMS 
frontend, which creates an overlay pool of grid resources.  In this architecture, every effort is given to find 
resources for the user (local, across campus, or externally), while maintaining the same vanilla Condor interface.

It is important to note the ever-widening circle of resources expands from the locality of the user.  It goes from the resource 
the user knows best (and has the best support for) to the most foreign one.  This is a very natural progression, and 
each step described comes with more complexity, and new failure modes.  If the user is ever frustrated at one 
transition, he can just remain contented with the resources he has, as opposed to having to switch between ``local 
mode" and ``grid mode".  Another usability boost is that all end-user interfaces are Condor vanilla universe.  The user 
never encounters errors translated between systems (a common user frustration) and the user needs to develop expertise 
in Condor alone.





\chapter{Evaluation}
\label{sec:Evaluation}

\subsubsection*{Social political stuff}

\subsubsection*{How we tested our solution}
\begin{itemize}
%\item   Performance metrics 
%\item   Performance parameters
%\item   Experimental design

\item Production jobs on Firefly

\end{itemize}


\subsubsection*{How our solution performed, how its performance compared to
that of other solutions mentioned in related work, and how these results show
that our solution is effective}

\begin{itemize}
%\item   Presentation and Interpretation
%\item   Why, how, and to what degree our solution is better
%\item   Why the reader should be impressed with our solution
%\item   Comments

\item
Faster job submission and start than globus.  Should be easy to show.

\item
Run at other campuses.

\item
Submit jobs between campuses.

\end{itemize}


\subsubsection*{Context and limitations of our solution as required for 
summation}
\begin{itemize}
%\item   What the results {\em do} and {\em do not} say
\item
Therefore, my solution is awesome.

\end{itemize}



\chapter{Conclusions and Future Work}
\label{sec:Conclusion}

\subsubsection*{The problem we have solved}
\begin{itemize}

%\item   The most succinct statement of the problem in the paper. Ideally one
%sentence. More realistically two or three. Remember that you simply state it
%without argument. If you have written a good paper you are simply reminding the
%reader of what they now believe and of how much they agree with you.

\item
The framework that I have described here creates a grid of clusters that can overflow to each other, and out to the other grids such as the Open Science Grid, or other campuses.  

\end{itemize}



\subsubsection*{Our solution to the problem}
\begin{itemize}

%\item    
%Again, the succinct statement that you have presented a solution

%\item   Sometimes it works well to leave it at that and not even describe your
%solution here. If you do, then again state your solution in one or two
%sentences taking the rhetorical stance that this is all obvious. If you have a
%good solution and have written an effective paper, then the reader already
%agrees with you.

\item
Combination of Condor, Glidein, and BLAHP.

\end{itemize}



\subsubsection*{Why our solution is worthwhile in some significant way}
\begin{itemize}

%\item   
%Again, a succinct restatement in just a few sentences of why your solution is
%worthwhile assuming the reader already agrees with you

\item
This solution keeps the administrator in control of priorities and access policies.  It can also follow the campus' security model.

\end{itemize}



%Why the reader should be impressed and/or pleased to have read the paper
\begin{itemize}

%\item   A few sentences about why your solution is valuable, and thus why the
%reader should be glad to have read the paper and why they should be glad you
%did this work.

\item
Provides a cookie cutter, well packaged solution for campuses to deploy a campus grid.

\end{itemize}



\subsubsection*{What we will (or could) do next}
\begin{itemize}
%\item   Improve our solution
%\item   Apply our solution to harder or more realistic versions of this problem
%\item   Apply our solution or a related solution to a related problem

\item 
Scale this solution to other universities and institutions.  Already doing this with the OSG Campus Grid Initiative.

\end{itemize}


\backmatter

\bibliographystyle{plain}
\bibliography{DerekWeitzelThesis}



\end{document}

\endinput

