\documentclass[11pt]{article}
\usepackage{times}

\setlength{\textwidth}{6.5in}
\setlength{\textheight}{9.0in}
\setlength{\topmargin}{-.5in}
\setlength{\oddsidemargin}{-.0600in}
\setlength{\evensidemargin}{.0625in}

\newcommand{\secref}[1]{Section~\ref{#1}}

\newcommand{\doublespace}{\baselineskip0.34truein}
\newcommand{\singlespace}{\baselineskip0.16truein}
\newcommand{\midspace}{\baselineskip0.24truein}
\newcommand{\midplusspace}{\baselineskip0.26truein}

\title{\bf Campus Grids}

\author{Derek Weitzel\\
Computer Science and Engineering\\
University of Nebraska--Lincoln\\
Lincoln, NE 66588-0115\\
dweitzel@cse.unl.edu
       }
       
\begin{document}

\maketitle

\begin{abstract}

At many universities, each department maintains a independent cluster.  Many of these clusters run under utilized.  In this paper we will describe a framework and technologies that where used to link departmental clusters such that submission at one cluster could lead to execution on another.  This framework is then further expanded to operating on a production national grid.

\end{abstract}

%   \tableofcontents
%   \newpage

% \doublespace

\section{Introduction}
\label{sec:Introduction}

\subsubsection* {The problem we have solved}

\begin{itemize}
%\item
%Concentrate on making {\em this} assertion and {\em only} this assertion in a
%succinct set of 1 to 3 paragraphs 

\item Department clusters waste power by being under utilized significant portions of time.
\item Researchers have peaks in usage (think paper writing) and need overflow capacity.
\item Overflow capacity for their local cluster.

%\item
%A common mistake is to explain too much of the problem context first. Instead,
%state the problem essentially as a claim, and leave explanations supporting
%your claim to the next part, ``Why it is not already solved.''

\end{itemize}


\subsubsection* {Why the problem is not already solved or other solutions 
are ineffective in one or more important ways}

\begin{itemize}
%\item
%Your new idea need not solve every problem but it should solve at least one
%that is not already solved

\item 
A common solution to linking clusters is condor flocking.  Flocking requires every cluster to run condor daemons on their nodes.  

\item
Another solution is single vendor/software solution.  This again requires buy-in from all departments on campus.

\item
Placing globus gatekeepers on each cluster would allow jobs to be submitted to each cluster without modifying the underlying batch system.  But, this would require a higher layer of abstraction over the globus gatekeepers to optimally balance load between clusters.  Also, it does not provide a method for transparent execution on other clusters.

%\item
%This is the place to provide a succinct description of the problem context
%giving enough information to support the claim that a problem exists, made in
%the preceding problem declaration.


\end{itemize}


\subsubsection* {Why our solution is worth considering and why is it effective
in some way that others are not}

\begin{itemize}
%\item
%A succinct statement of {\em why} the reader should care enough to read the
%rest of the paper.
\item The framework described in this paper is designed as a modular framework that will allow clusters overflow onto each other and to the grid.  This framework requires running condor on only one node in the cluster.  Each job that comes from another cluster will go through the default scheduler, whether it's PBS, SGE, or LSF.

%\item
%This should include a statement about the characteristics of your solution to
%the problem which 1) make it a solution, and 2) make it superior to other
%solutions to the same problem.

\end{itemize}


\subsubsection* {How the rest of the paper is structured}
\begin{itemize}
\item
The short statement below is often all you need, but you should change it when
your paper has a different structure, or when more information is {\em
required} to describe what a given section contains. If it isn't {\em required}
then you don't want to say it here.

\end{itemize}

The rest of this paper first discusses related work in
\secref{sec:RelatedWork}, and then describes our implementation in
\secref{sec:Implementation}. \secref{sec:Evaluation} describes how we evaluated
our system and presents the results. \secref{sec:Conclusion} presents our
conclusions and describes future work.


\section{Related Work}
\label{sec:RelatedWork}

\subsubsection*{Other efforts that exist to solve this problem and why are they
less effective than our method}

\begin{itemize}
%\item
%Resist the urge to point out only flaws in other work. Do your best to point
%out both the strengths and weaknesses to provide as well rounded a view of how
%your idea relates to other work as possible
\item
Globus is a translation layer between the globus specific Resource Specification Language, and the local resource manager.  It has been very successful in that is has a large install base.  Globus also has deep integration with standard grid credentials.

\item
Condor Flocking



%\item
%In a social and political sense, it is {\em very smart} as well as ethical to
%say good things, which are true, about other people's work. A major motivation
%for this is that editors and program committee members have to get a set of
%reviews for your paper. The easiest way for them to decide who should review it
%is to look at the set of references to {\em related work} (e.g.,
%\cite{ARJ:95,BHR:90,Go:97}) to find people who are likely to be competent to
%review your paper.  The people whose work you talk about are thus likely to be
%reading what you say about {\em their} work while deciding what to say about
%{\em your} work. 

%\item
%Clear enough? Speak the truth, say what you have to say, but be generous to the
%efforts of others.

\end{itemize}


\subsubsection*{Other efforts that exist to solve related problems that are
relevant, how are they relevant, and why are they less effective than our
solution for this problem}

\begin{itemize}

\item
Diagrid

\item 
GlideinWMS

\item
Panda

%\item 
%Many times no one has solved your exact problem before, but others have solved
%closely related problems or problems with aspects that are strongly analogous
%to aspects of your problem

\end{itemize}

\section{Implementation}
\label{sec:Implementation}

\subsubsection*{What we (will do $|$ did): {\em Our Solution}}
\begin{itemize}

\item Glideins

\item   Another way to look at this section is as a paper, within a paper,
describing your implementation. That viewpoint makes this the introduction to
the subordinate paper, which should describe the overall structure of your
implementation and how it is designed to address the problem effectively.

\item   Then, describe the structure of the rest of this section, and what each
subsection describes.

\end{itemize}


\subsubsection*{How our solution (will $|$ does) work}
\begin{itemize}
\item   This is the body of the subordinate paper describing your solution. It
may be divided into several subsections as required by the nature of your
implementation.

\item   The level of detail about how the solution works is determined by what
is appropriate to the type of paper (conference, journal, technical report)

\item   This section can be fairly short for conference papers, fairly long for
journal papers, or {\em quite} long in technical reports. It all depends on the
purpose of the paper and the target audience

\item   Proposals are necessarily a good deal more vague in this section since
you have to convince someone you know enough to have a good chance of building
a solution, but that you have not {\em already} done so.

\end{itemize}


\section{Evaluation}
\label{sec:Evaluation}

\subsubsection*{How we tested our solution}
\begin{itemize}
\item   Performance metrics 
\item   Performance parameters
\item   Experimental design

\end{itemize}


\subsubsection*{How our solution performed, how its performance compared to
that of other solutions mentioned in related work, and how these results show
that our solution is effective}

\begin{itemize}
\item   Presentation and Interpretation
\item   Why, how, and to what degree our solution is better
\item   Why the reader should be impressed with our solution
\item   Comments

\end{itemize}


\subsubsection*{Context and limitations of our solution as required for 
summation}
\begin{itemize}
\item   What the results {\em do} and {\em do not} say

\end{itemize}



\section{Conclusions and Future Work}
\label{sec:Conclusion}

\subsubsection*{The problem we have solved}
\begin{itemize}
\item   The most succinct statement of the problem in the paper. Ideally one
sentence. More realistically two or three. Remember that you simply state it
without argument. If you have written a good paper you are simply reminding the
reader of what they now believe and of how much they agree with you.

\end{itemize}



\subsubsection*{Our solution to the problem}
\begin{itemize}
\item    
Again, the succinct statement that you have presented a solution

\item   Sometimes it works well to leave it at that and not even describe your
solution here. If you do, then again state your solution in one or two
sentences taking the rhetorical stance that this is all obvious. If you have a
good solution and have written an effective paper, then the reader already
agrees with you.

\end{itemize}



\subsubsection*{Why our solution is worthwhile in some significant way}
\begin{itemize}
\item   
Again, a succinct restatement in just a few sentences of why your solution is
worthwhile assuming the reader already agrees with you

\end{itemize}



Why the reader should be impressed and/or pleased to have read the paper
\begin{itemize}
\item   A few sentences about why your solution is valuable, and thus why the
reader should be glad to have read the paper and why they should be glad you
did this work.

\end{itemize}



\subsubsection*{What we will (or could) do next}
\begin{itemize}
\item   Improve our solution
\item   Apply our solution to harder or more realistic versions of this problem
\item   Apply our solution or a related solution to a related problem

\end{itemize}




\bibliographystyle{plain}
\begin{thebibliography}{99}%\setlength{\itemsep}{-1ex}
% \small
\singlespace

\bibitem{ARJ:95} Anderson, J., Ramamurthy, S., Jeffay, K.,
``Real-Time Computing with Lock-Free Shared Objects''\negthinspace,
{\em Proceedings of the 16th IEEE Real-Time Systems Symposium\/},
IEEE Computer Society Press, December 1995, pp.\ 28-37.

\bibitem{BHR:90} Baruah, S., Howell, R., Rosier, L.,
``Algorithms and Complexity Concerning the Preemptively Scheduling of Periodic,
Real-Time Tasks on One Processor,'' {\em Real-Time Systems Journal\/},
Vol. 2, 1990, pp. 301-324.


\bibitem{Go:97} Goddard, S., Jeffay, K. ``Analyzing the Real-Time Properties of
a Dataflow Execution Paradigm using a Synthetic Aperture Radar Application,''
{\em Proc. IEEE Real-Time Technology and Applications
Symposium\/}, June 1997, pp.\ 60-71.

%\bibliography{/usr/users/niehaus/thesis/bib/papers/all}

\end{thebibliography}


\end{document}
